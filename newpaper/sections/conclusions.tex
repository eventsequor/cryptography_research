The comparative analysis of various random number generators, including classical pseudo-random (CRNG), Counter Mode Deterministic Random Bit Generators (CTR DRBG and CTR DRBG with sequential seeding), and a Quantum Random Number Generator (QRNG), was conducted using a selection of five statistical tests from the NIST SP 800-22 Rev 1a suite. These tests, namely the Frequency (Monobit) Test, Frequency Test within a Block, Runs Test, Maurer's "Universal Statistical" Test, and the Cumulative Sums (Cusum) Test, proved invaluable in assessing different aspects of randomness, from basic bit balance to compressibility and trend analysis . The results indicated that both CTR DRBG implementations exhibited excellent statistical properties, passing all applied tests and demonstrating characteristics consistent with high-quality random data suitable for cryptographic applications. In contrast, the evaluated QRNG, despite its theoretical underpinnings for true randomness, showed significant deviations, failing critical tests such as the Monobit, Runs, and Cusum tests, suggesting underlying biases or issues in the generation or collection process that compromise its statistical randomness for the tested 500,000-bit sequence. The classical RNG served as a baseline and generally performed well.
