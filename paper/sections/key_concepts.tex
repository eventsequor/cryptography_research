Most RNG methods can be classified into one of two categories. The first of these is Pseudo Random Number Generation, sometimes called Deterministic Random Bit Generation (DRBG) (Cao et al., 2022). A PRNG is a deterministic algorithm that produces numbers that appear to be random. It requires a seed value to generate a seemingly random number, and due to its deterministic nature, the same seed will produce the same value every time. PRNGs experience periodicity, for after exhausting all possible internal variations, it will repeat cycles that will reiterate on the sequences of produced numbers. Good PRNG algorithms, however, manage to display good statistical behaviors, with some having periods in order of magnitudes so large they become negligible. Due to their algorithmic nature, PRNGs are quick and scalable; their deterministic nature is also desirable in experiments where replicability is key. However, they are extremely unsafe key producers for cryptographic measures, for a backtrack of the algorithm or the knowledge of the seed reveal the output in its entirety  (Mechalas, 2018). 
True Random Number Generators (TRNG), on the other hand, aim to produce true random values. To achieve this, TRNG relies on entropy sources, which extract true randomness by extracting information from physical phenomena. The two main types of entropy sources are dynamic entropy sources, which extract true random values from indeterminate physical processes like thermal noise or atmospheric noise, and static entropic sources, which extract randomness from randomly occurring properties in the hardware components of the computer as a result of the semiconductor manufacturing process, which become stable once the device is finished. These properties can be found in chip and are used for things like authentications (Cao et al., 2022). What is essential is that an entropy source extracts its randomness from the physical world, which means that no “randomness” generated by a procedural method can be considered an entropy source. TRNGs are desirable where safety is essential, and show constant distributions that guarantee unpredictability, but are usually slow to output a number due to their need to measure physical phenomena; this takes time and is computationally costly, which makes them not very scalable.
Some methods can be classified into particular categories of RNG.  Cascade Construction Random Number Generator (CCRNG), for example, relies on an entropy source to supply an “entropy buffer”, which is then used to provide cryptographically secure PRNG. Digital Random Number Generators (DRNGs) is an approach that builds the RNG on the processor’s hardware directly, and uses a combination of CCRNG and dynamic entropy sources to create random streams. Even so, this categories are usually some sort of combination of PRNG and TRNG to different degrees, and combining both methods is becoming more common in the industry to tackle the strengths and weaknesses of both methods.
