\section{Methology}
\subsection{Random Number Generation}
\label{sec:random_generation}

For this exercise, three different forms of random number generation were performed. The first was through the \textit{CTR\_DRBG} algorithm, which used random keys generated by the \texttt{random} function of the Python programming language version 3.9 as seeds and was executed on a MacBook Pro M3 Pro computer. This detail may lead to different results if the experiment is replicated. However, we assume that Python's \texttt{random} function generates a uniform randomness distribution, simulating an entropic system for generating random bits.

For the second option, a sequential range of seeds was chosen, which were formatted in 32 bits to ensure compatibility with AES encryption. The sequence started at 1 and increased by one until reaching 50 million seeds. These seeds were then used to execute the \textit{CTR\_DRBG} algorithm, which generates the pseudo-random bit values.

The third approach made use of the Beacon system. In this case, the official NIST API was consumed to retrieve the last 15,600 random bits, generated from 512-bit values, obtained from the historical records maintained by the system.
